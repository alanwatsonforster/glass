\section*{F4U and AU Corsair}

The F4U Corsair fighter was designed and built by Vought for the USN. It featured superb performance, but its long nose and a cockpit well to the rear meant it was a challenge to land on an aircraft carrier. During its long gestation, it was employed as a land-based fighter with the USMC, but by the end of WWII, it was regarded as the best carrier-based fighter in service. By the time of the Korean War, it had been replaced as a day fighter by the jet-engined F9F Panther, but continued to serve as a fighter-bomber in the USN and USMC.

The F4U-4 is the last version that was constructed during WWII and maintains the original armament of six .50 cal M2 machine guns. The F4U-4B is basically a -4 with four 20 mm M3 cannon substituting the machine guns. These two versions were used in large numbers in the Korean War for close air support and interdiction, both by the USN and USMC. The -4 was preferred for carrier operations, as its guns were easier to service in the confined spaces of the hanger deck of an aircraft carrier, and the -4B tended to be used by land-based USMC squadrons.

The F4U-5 is a post-WWII version with many refinements based on experience with the -4 and maintaining the 20~mm armament of the -4B. For reasons that are not clear to me, it did not see service in the Korean War.

The AU-1 is a dedicated close air support aircraft for the USMC, derived from the FU4-5, but with heavier armor, a simpler supercharger designed for operations at lower altitudes, and additional weapons stations. It entered service in 1952 and saw combat in the Korean War.

The gun armament of the -4 is six .50 cal M2 machine guns with about 400 rounds per gun (400 rounds for the inner two and 375 rounds for the outer one). The -4B, -5, and AU-1 have four 20 mm M3 cannon with 231 rounds per gun. 

A typical air-to-ground armament for the -4 and -4B in the Korean War would be TODO. 

ADCs are provided for the:
\begin{itemize}
\item F4U-4
\item F4U-4B
\item F4U-4P
\item F4U-5
\item F4U-5P
\item F4U-5N
\item F4U-5NL
\item AU-1
\end{itemize}
