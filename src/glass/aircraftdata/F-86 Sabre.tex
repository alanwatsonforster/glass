\section*{F-86 Sabre}

The North American F-86 Sabre is a single-engined, transsonic fighter, interceptor, fighter-bomber, and tactical nuclear bomber. It was the first swept-wing aircraft to enter service in the USAF. There are many versions and variants of the F-86, split first into the fighter F-86A/E/F/H series and the interceptor F-86D/K/L series and then by engine, provision for stores and armament, and wings. Here we discuss first the fighters and then the interceptors.

The F-86A has a swept wing and a swept tail with conventional control surfaces. It is powered by a J47-GE-7 engine with 5340 lb of dry thrust; only the interceptor versions of the F-86 had afterburners. It is armed with six .50 cal M3 machine guns and the view from the cockpit is excellent. Like many of its contemporaries, it suffers from a very short range on internal fuel, and relied on a pair of either 120 gal or 206 gal fuel tanks on under-wing stations to achieve a useful range. However, the 206 gal tank limits maneuvers to 3G and so can only be used for ferry flights. Instead of fuel tanks, it can carry two bombs of up to 1000 lb or sixteen 5-inch HVAR rockets, but has a very restricted combat range while doing so, and therefore served essentially as a day fighter rather than a fighter bomber. It entered service with the USAF in 1949. It was rushed to Korea to counter Soviet MiG-15s, which were outclassing the straight-wing F-80C fighters already in theater, and saw combat starting in December 1950.

The F-86E improves on the F-86A by having a slightly more powerful J47-GE-13 engine with 5450 lb of dry thrust, the A-1CM radar-ranging gunsight, and an all-flying tail. The new tail significantly improves the handling compared to the F-86A at high transonic speeds. It entered service in February 1951 and arrived in Korea in July 1951.

The F-86F series introduces a series of significant improvements over earlier models. All F-86F variants are powered by the J47-GE-27 engine, producing 5970 lb of dry thrust. The initial F-1 block otherwise closely resembles the F-86E. It entered service in April 1952 and arrived in Korea by June 1952. The F-5 block adds the improved A-4 gunsight. The F-10, F-15, and F-20 blocks permit the use of 200 US gal fuel tanks without maneuver restrictions, greatly enhancing operational range for day fighter missions. These blocks entered service starting in June 1952.

The F-25 and F-30 blocks hugely improve the capability of the F-86 as a fighter bomber. They have a second pair of weapon pylons on the inner wing, allowing the aircraft to carry both a pair of bombs or eight HVAR rockets in addition to a pair of fuel tanks. This configuration provides a much improved range when carrying air-to-ground weapons. The F-25 models entered service in Korea in October 1952. 

The F-35 block is specialized as a tactical nuclear bomber, with a strengthened the left inner pylon to permit carriage of the second-generation Mk 12 nuclear bomb and the LABS (Low Altitude Bombing System) to allow toss bombing. It entered service in January 1954 and was based mainly in Europe.

All the versions mentioned to here are equipped with the basic wing with automatic leading-edge slats that increased lift at low speed. In the summer of 1952, North American developed the “6-3 wing” or the “solid 6-3 wing”, which features a 6-inch extension to the chord at the wing root and a 3-inch extension at the wing tip. The resulting larger area gives much better maneuverability at high altitude. The cost is the loss of the automatic slats, which increases the landing speed and the landing roll, and led to accidents when used by inadequately-trained pilots. Starting in September 1952, field-modification kits were provided to add the 6-3 wing to existing F models and later-production F-25/30/35 had them installed in the factory.

The F-86H further refines the F-86 as a fighter bomber. It has a more powerful, albeit heavier, J73-GE-3 engine producing 8900 lb of dry thrust and giving a much improved rate of climb. The initial H-1 block maintains the armament of six 0.50 cal machine guns, but the later H-5/10 blocks have four 20 mm M39 guns each with 150 rounds, finally giving the Sabre a potent air-to-air armament, albeit in a model not intended as a fighter. It entered service in 1954.

The F-86F-40 then appeared initially as a version to be manufactured for the JASDF by Mitsubishi under license. It is similar to the F-25, but features a new wing with 12-inch span extensions at the wing-tips and once more automatic leading-edge slats. As such, it combined the superior high-altitude maneuverability of the 6-3 wing with the lower landing speeds of the basic wing; it was known as the “extended 6-3 wing” or just the “F-40 wing”. This version was also built for the USAF and supplied to many US allies under the MAP (Military Assistance Program), with deliveries starting in 1955. Many earlier F-series aircraft were upgraded to the F-40 standard and the extended 6-3 wing was also refitted to the F-86H, F-86L, and F-86K versions.

The interceptor F-86D/K/L versions of the F-86 are significantly different to the F-86A/E/F/H versions, sharing wings and tail but little else. Informally, they were known as the “Sabre Dog”.

The F-86D has the J47-GE-17 engine with 5500 lb of dry thrust and an afterburner giving 7650 lb; the afterburner was vital to allow the aircraft to quickly climb to the high altitudes at which it was expected to intercept intruding bombers. The larger fuselage allows more internal fuel, but despite this the F-86D and subsequent interceptors almost always flew with two 120 gal fuel tanks on wing stations. The F-86D has the basic wing and all-flying tail of the F-86E. Its weapon system consisted of an APG-7 radar, twenty-four 70 mm FFAR rockets in an extending ventral tray, and a Hughes E-4 fire-control system linking the two. The combination permitted collision-course rocket attacks from the beam, but presented a heavy workload for the single pilot. Early variants of the aircraft began to be delivered in March 1951, but versions with the E-4 fire control system were only delivered starting in July 1952.

The F-86K is a simplified interceptor developed from F-86D for export under the Military Assistance Program (MAP). The weapon system was reworked, with four 20 mm M24 guns replacing the FFARs and a Hughes MG-4 fire-control system replacing the more advanced E-4 system. The simpler fire-control system no longer had collision-course attacks was capable of tail-chase attacks. Many F-86Ks for European air forces were assembled by Fiat from kits provided by North American; others were built by North American itself. Some F-86Ks have the extended 6-3 wing of the F-40, either as late-production builds or as upgrades. 

The F-86L is a rebuild of the F-86D. It has the extended 6-3 wing and updated electronics, including a data link to allow the  intercepts to be directly guided by the computerized SAGE system. It entered service in late 1957.

From 1958 onwards, some F-86D, F-86K, and F-86L aircraft were retrofitted with two missile stations inboard of the fuel tanks for the Sidewinder missile. This upgrade enhanced their air-to-air combat capability, particularly against more maneuverable or heavily armed adversaries.


ADCs are provided for:
\begin{itemize}
    \item F-86A
    \item F-86D
    \item F-86E
    \item F-86F-1
    \item F-86F-1 (6-3 Wing)
    \item F-86F-5
    \item F-86F-5 (6-3 Wing)
    \item F-86F-10
    \item F-86F-10 (6-3 Wing)
    \item F-86F-25
    \item F-86F-25 (6-3 Wing)
    \item F-86F-35
    \item F-86F-35 (6-3 Wing)
    \item F-86F-40
    \item F-86H
    \item F-86H (Extended 6-3 Wing)
    \item F-86K
    \item F-86K (Extended 6-3 Wing)
    \item F-86L
\end{itemize}