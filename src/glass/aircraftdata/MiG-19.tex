\section*{Mikoyan-Gurevich MiG-19}

The Mikoyan-Gurevich MiG-19 is a day fighter and interceptor. It is the first Soviet production aircraft capable of supersonic speed in level flight. It is a significantly different design to its predecessor, the MiG-17, as it has twin Mikulin RD-9 engines. Like most of its contemporaries, it suffers from short range on internal fuel, and this is mitigated by the use of underwing drop tanks.

The initial MiG-19 (Farmer-A) day fighter entered service in 1955 with the VVS. It is the only version with a conventional (not an all-flying) tail. It is armed with three 23\,mm NR-23 cannons, two in the wings and one under the nose. This was considered an improvement over the mixture of two 23\,mm NR-23 and one 37\,mm N-37 cannon in the MiG-15 and MiG-17, as these different caliber guns have very different ballistic characteristics and so the shell streams diverge at long range

The MiG-19S (Farmer-C) is an improved version of the MiG-19. It incorporated an all-flying tail and updated avionics. The early versions maintained the armament of three 23\,mm NR-23 guns, but later versions switched to three 30\,mm NR-30 guns. The NR-30 has slightly better rate of fire and muzzle velocity compared to the NR-23, but fires a much heavier shell. To a large degree, it combines the dogfight characteristics of the NR-23 with the anti-bomber characteristics of the N-37. The MiG-19SF has slightly more powerful engines, but is otherwise very similar to the late MiG-19S.

The MiG-19P (Farmer-B) is an all-weather interceptor. It is based on the MiG-19S, but has a modified nose and cockpit incorporating the RP-1 Izumrud (Scan Odd) radar and has only the guns in the wing roots. Again, the early versions have two 23\,mm NR-23 guns, but later versions have two 30\,mm NR-30 guns. In addition to underwing fuel tanks, it often carried two underwing rocket pods for use against bombers.

The MiG-19PM (Farmer-E) is also an all-weather interceptor. It is essentially a MiG-19P with the guns removed, provision for four Kaliningrad K-5 (AA-1 Alkali) beam-riding missiles, and the radar modified to support the missiles.

The MiG-19S was built under license in Czechoslovakia as the Aero S-105. The MiG-19S, MiG-19P, and MiG-19PM were built under license and further developed in China as the Shenyang J-6/F-6.

The MiG-19 saw service in the VVS, PVO, and AVMF starting in 1955, and later with Soviet allies including Afghanistan, Albania, Bulgaria, China, Cuba, Czechoslovakia, East Germany, Egypt, Hungary, Indonesia, Iraq, Pakistan, Poland, North Korea, Romania, and Syria. In some cases, these MiG-19s served alongside Shenyang F-6s.

In Soviet service it unsuccessfully attempted to intercept U-2 overflights. Egyptian and Syrian MiG-19s saw combat in the 1967 War. In Egyptian service, it saw combat in the 1967 War and the War of Attrition.

ADCs are provided for:
\begin{itemize}
    \item MiG-19S (Early)
    \item MiG-19S (Late)
    \item MiG-19P (Early)
    \item MiG-19P (Late)
    \item MiG-19PM
\end{itemize}