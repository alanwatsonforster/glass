\section*{Tupolev Tu-16}

The Tupolev Tu-16 is a conventional and nuclear strategic bomber. It has a swept wing and tail and two large Mikulin AM-3 engines in the wing roots. It is defended by six 23 mm AM-23 guns mounted in pairs in a tail turret, rear ventral turret, and forward dorsal turret, and also has a fixed forward-firing single 23 mm AM-23 gun.

The initial Tu-16 version is a conventional strategic bomber, and was the Soviet Union's first long-range jet bomber.

The Tu-16A is an adaptation of the Tu-16 for carrying nuclear weapons, including the Soviet Union's first hydrogen bomb, the RDS-37.

The Tu-16KS and Tu-16K were naval strike version, with improved search radar and the ability to carry KS-1 Komet (AS-1 Kennel) and KSR-2/KSR-11 (AS-5 Kelt) cruise missiles.

The Tu-16 entered service in 1954 with DA (Long-Range Aviation) and AVMF (Naval Aviation). The Tu-16A followed shortly thereafter. The Tu-16KS and Tu-16K entered service in 1954 and 1962 with AVMF.

The Tu-16 was exported to China (where is was also produced under license as the Xi'an H-6/B-6), Egypt, Indonesia, and Iraq.

Egyptian Tu-16s suffered heavy losses on the ground at the start of the 1967 War, but were more active in the 1973 War.

Iraqi Tu-16s saw combat in the Iran-Iraq War.

ADCs are provided for:
\begin{itemize}
    \item Tu-16 (Badger-A)
    \item Tu-16A (Badger-A)
    \item Tu-16KS (Badger-B)
    \item Tu-16K (Badger-G)
\end{itemize}