\section*{Mikoyan-Gurevich MiG-17}

The Mikoyan-Gurevich MiG-17 is an early jet interceptor and day fighter, with a limited capability as a fighter-bomber. It is based on the MiG-15bis, but has a thinner but stiffer wing and tail surfaces for better control at higher Mach number and a longer fuselage, but retained conventional elevators. It also continued with the same heavy armament as the MiG-15bis, one 37 mm N-37 and two 23 mm NR-23 cannons, appropriate for its primary role as an interceptor.

The original MiG-17 (Fresco-A) day interceptor is powered by the Klimov VK-1A non-afterburning engine, similar to that of the MiG-15bis, and entered service in 1952. 

The MiG-17P (Fresco-B) all-weather interceptor is derived from the original MiG-17. It has an RP-1 Izumrud (Scan Odd) radar mounted in front of the air intake.

The MiG-17F (Fresco-C) day fighter was introduced in 1953. It has a VK-1F afterburning engine but was otherwise almost identical to the MiG-17. The afterburner gave a dramatic improvement in climb rate, which is important in an interceptor. The MiG.-17-F was the first Soviet operational aircraft to be equipped with an afterburner. The Polish PZL-Mielec Lim-5 and Chinese Shenyang J-5/F-5 are licensed versions of the MiG-17F.

The MiG-17PF (Fresco-D) all-weather interceptor is a development of the MiG-17P. It gains the VK-1F afterburning engine and has an armament of three NR-23 guns. The Polish PZL-Mielec Lim-5P and Chinese Shenyang J-5A/F-5A are licensed versions of the MiG-17PF.

The MiG-17PFU (Fresco-E) all-weather interceptor is a further development of the MiG-17PF. The major change is the replacement of the gun armament with four Kalingrad K-5 (AA-1 Alkali) BRMs guided by the RP-2 Izumrud radar.

The MiG-17 served in the VVS, PVO, and Naval Aviation until the development of faster Western bombers such as the B-47 and the V-bombers rendered it obsolete. It was widely exported to Warsaw Pact countries, China, and other allies of the Soviet Union.

Chinese PLAAF MiG-17s clashed with RoCAF F-86s in 1958.

The MiG-17/17F famously saw combat in the Vietnam War, embarrassing much more modern adversaries with its small size, maneuverability, and cannon armament. 

Egyptian and Syrian MiG-17/17F aircraft served in a ground-attack role in the 1967 War, the War of Attrition, and the 1973 War.

ADCs are provided for:
\begin{itemize}
    \item MiG-17
    \item MiG-17F
    \item MiG-17P
    \item MiG-17PF
    \item MiG-17PFU
\end{itemize}