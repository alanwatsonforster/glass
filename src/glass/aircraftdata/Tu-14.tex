\section*{Tupolev Tu-14}

The Tupolev Tu-14 is a conventional medium bomber and torpedo bomber. Its NATO reporting name is Bosun. It was developed in competition with the Ilyushin Il-28, and shares many features with that aircraft including unswept wings, a swept tail, two Klimov VK-1 engines in pods under the wings, and a gun armament of two fixed 23 mm NR-23 guns and two more in a tail turret.

The Tu-14 is a conventional tactical bomber. It can carry 3,000 kg (6,600 lb) in its internal bomb bay. However, it was not accepted for service by the VVS, which preferred the Il-28.

The Tu-14T is a torpedo bomber, developed for Naval Aviation. It can carry torpedoes, mines, or bombs in its internal bomb bay. In contrast to the Tu-14 bomber version, the pilot and weapons officer are provided with ejection seats.

The Tu-14T served from 1952 to 1959 in Naval Aviation, but did not serve in other branches and was not exported.

ADCs are provided for:
\begin{itemize}
    \item Tu-14
    \item Tu-14T
\end{itemize}