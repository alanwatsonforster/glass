\section*{Mikoyan-Gurevich MiG-15}

The Mikoyan-Gurevich MiG-15 is an early jet interceptor and day fighter. The MiG-15 was one of the first fighters to incorporate a swept wing and a swept tail to reduce the effects of compressibility and allow higher transonic speeds. However, it used conventional elevators rather than an all-flying tail, and so became difficult to control at high speeds. The NATO reporting name is “Fagot.”

The initial MiG-15 was powered by the Klimov RD-45, an unauthorized copy of the Rolls-Royce Nene engine (which had been sold to the Soviet Union in small numbers with the agreement of the British Government). As it was initially intended to serve as an interceptor, it carried a very heavy gun armament of one 37 mm N-37 cannon with 40 rounds and two 23 mm NS-23 cannon with 80 rounds per gun, all under the nose. The NATO reporting name is “Fagot-A.”

The MiG-15bis was the main production version and was an improvement on the original MiG-15 in a number of small but important ways. The engine was upgraded to the Klimov VK-1, a development of the RD45 with more power. The NS-23 cannons were replaced by faster-firing NR-23 cannons. The NATO reporting name is “Fagot-B.”

The MiG-15P was a prototype all-weather interceptor, equipped with the RP-1 Izumrud radar (NATO reporting name “Scan Fix”) and with its armament reduced to two 23 mm NR-23 cannons.

The MiG-15ISh is a prototype attack version with pylons in the wings for bombs or rockets in addition to the stations for fuel tanks.

The MiG-15 and MiG-15bis served with the Soviet VVS and PVO and in the air forces of many Soviet allies. They saw extensive combat, including in the Korean War with the VVS and PVO, the PLAAF, and the KPAF, and in clashes between the PLAAF and RoCAF, in the Suez Crisis with the EAF. They also performed many interceptions of aircraft during the Cold War. When used as a fighter, it was limited by the slow rate of fire of its guns (400 rounds per minute for the N-37 and 800 for the NR-23), but a single hit on a fighter or fighter-bomber could often inflict fatal damage. As an interceptor in Korea, its combination of high speed and powerful armament allowed it to inflict heavy losses of USAF B-29 bombers engaged in daytime raids and effectively force them to operate under the cover of darkness.

The MiG-15 formed the basis for the development of the MiG-17.

ADCs are provided for:
\begin{itemize}
    \item MiG-15bis
    \item MiG-15P
    \item Mig-15ISh
\end{itemize}
