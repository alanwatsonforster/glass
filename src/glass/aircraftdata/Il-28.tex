\section*{Ilyushin Il-28}

The Ilyushin Il-28 is a conventional and nuclear tactical bomber and a long-range reconnaissance aircraft. The NATO reporting name is “Beagle”.

It is powered by two Klimov VK-1 engines in underwing pods. This engine is a development on the Klimov RD-45 unauthorized copy of the Rolls-Royce Nene engine, and was also used in the MiG-15bis.

The Il-28 is a conventional tactical bomber. It is armed with two fixed 23 mm NR-23 cannons and two more in a defensive tail mount. It can carry up to 3,000 kg (6,600 lb) of bombs in its internal bomb bay, but a normal load is 1,000 kg (2,200 lb).

The Il-28R is a long-range photo-reconnaissance aircraft. Its bomb bay is given over to cameras, flares, and additional fuel. It also is equipped with two 350L wing-tip fuel tanks. One of the forward-firing guns is replaced by reconnaissance equipment.

The Il-28N is a nuclear tactical bomber. It can carry the RDS-4 nuclear bomb in its internal bomb bay. Like the Il-28R, it is equipped with two 350L wing-tip fuel tanks.

The Il-28 entered service in the VVS in 1950 and AVMF (Naval Aviation) in 1951. The IL-28R followed shortly thereafter in 1952. They were exported to many countries, including Afghanistan, Algeria, Bulgaria, Cambodia, Czechoslovakia, East Germany, Egypt, Finland, Hungary, Indonesia, Iraq, Morocco, Nigeria, North Korea, North Vietnam, Romania, Somalia, Syria, and Yemen. They were built under license in Czechoslovakia. China later built a modified version as the Harbin H-5.

The Il-28s entered service with the PLAAF in 1952, and was an implicit threat during the last year of the Korean War. They saw action in 1956 against Taiwan, bombing the Tachen Islands, and suffered losses to RoCAF F-84 and F-86s.

Egyptian Air Force Il-28s fought in the 1967 War, the War of Attrition, and the 1973 War.

ADCs are provided for:
\begin{itemize}
    \item Il-28
    \item Il-28R
    \item Il-28N
\end{itemize}
