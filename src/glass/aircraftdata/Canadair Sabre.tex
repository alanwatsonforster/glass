\section*{Canadair Sabre}

The Canadair Sabre is a day fighter derived from the North American F-86 Sabre. The initial versions were only lightly modified, but later versions incorporated the more powerful Orenda engine. The Sabre Mk.6 competes with the CAC Sabre Mk.32  for the honor of being the very best day-fighter Sabre.

The single Sabre Mk.1 was a prototype very similar to the F-86A. 

The first production version was the Mk.2, which was essentially a F-86E. The Mk.4 was very similar. Both of these versions were built with the original Sabre slatted wing. 

Sabre Mk.2s were used by the RCAF in Europe and the USAF. Later, they were then passed on to the air forces of Greece and Turkey. In USAF service, they saw combat in the Korean War.

Sabre Mk.4s were used in small numbers by the RCAF and in larger numbers by the RAF, where they were known as the Sabre F.4 and served alongside the Meteor F.8. They were the first swept-wing fighter in British service. Starting in 1954, they begin to be refitted with the 6-3 wing. As Hawker Hunters became available in 1956, the RAF Sabres were transferred to the Yugoslav and Italian air forces.

The Canadair Sabres began to diverge from the North America originals with the Mk.3 prototype, which used an Avro Canada Orenda 3 engine with significantly more thrust than that of the Mk.2. This prototype was subsequently developed into the production Mk.5 version with the improved Orenda 10 engine and the unslatted 6-3 wing. The last version was the Mk.6 with the even more powerful Orenda 14 engine. Later Mk.6s gained the slatted 6-3 wing. All the production versions were fitted with the original F-86 armament of six .50~cal M3 guns.

Sabre Mk.5s were initially used by the RCAF, again mainly in Europe, replacing the Mk.2s. A number were later transferred to the Luftwaffe.

The Mk.6s in turn replaced the RCAF Mk.5s and also were used in large numbers by the Luftwaffe. These were later sold on to the Columbia, South Africa, and Pakistan. In PAF service they were known, confusingly, as the F-86E and fought in the 1971 war with India.

A typical air-to-air load is two 200 gal (750L) FTs on the outer stations and, From 1960, on the Mk.6, two AIM-9B IRMs on the inner stations. For air-to-ground, two 1000 lb bombs can be carried on the inner stations along with two 200 FTs on the outer stations. Alternatively, on the later versions, sixteen HVAR rockets can be carried without fuel tanks. For ferry flights, two 120 gal (400L) FTs can be carried on the inner stations and two 200 gal (750L) FTs to the outer ones.

ADCs are provided for:
\begin{itemize}
\item Sabre Mk.2
\item Sabre Mk.4
\item Sabre Mk.4 (6-3 Wing) --- Sabre Mk.4 refitted with the unslatted 6-3 wing
\item Sabre Mk.5
\item Sabre Mk.6
\item Sabre Mk.6 (Slatted 6-3 Wing) --- Sabre Mk.6 refitted with the slatted 6-3 wing
\end{itemize}
