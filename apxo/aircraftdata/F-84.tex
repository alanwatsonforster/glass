\section*{Republic F-84 Thunderjet}

The Republic F-84 Thunderjet is an early escort fighter, fighter-bomber, and tactical nuclear bomber. It has unswept wings, a single jet engine, and an air intake in the nose. 

The aircraft was designated P-48 during its development. The P-84B entered service in 1947 and rapidly was redesignated as F-84B; there was no P-84A version, as the XP-84A was a service test version. It was armed with six .50~cal M3 machine guns, with four guns were installed in the nose and one in each wing root. It was equipped with an Allison J35-A-15 engine and, to improve its range, wing-tip fuel tanks with a capacity of 226 gal each. The F-84C followed shortly after, and was almost identical to the F-84B, with the main change being the use of the J35-A-13C engine, which had better reliability but otherwise had almost identical performance. Both of these models suffered from structural damage to the fuselage skin, which lead to performance limitations, and structural failures in the wings, caused at least in part by the fuel tanks applying unexpected twisting loads to the wings during high-g maneuvers.

The first satisfactory version was the F-84D, which entered service in 1949. It had strengthened wings, triangular fins on the wing-tip tanks to mitigate twisting, and a more powerful J35-A-17D engine.

The F-84E entered service in 1950 and represented a further improvement on the F-84D. It has lengthened fuselage, further strengthened wings, added an A-1B gun sight with an APG-30 range-finding radar, and had provision for 230~gal fuel tanks on the under-wing stations. The F-84E could be equipped with air-to-air refueling probes on each wing-tip fuel tank, and these were used in combat in Korea to permit long-range missions to be flown from Japan.

The final F-84G version was intended to bridge the gap before the introduction of the swept-wing F-84F. It was equipped with an air-to-air refueling receptacle in its left wing (in addition to possible use of air-to-air refueling probes in the wing-tip fuel tanks, like the F-84E), a new framed canopy as the single-piece canopy of the F-84E was subject to failure, and could carry a Mk~7 nuclear bomb. The new canopy was subsequently refitted to previous versions.

In the Korean War, when the MiG-15 entered combat on 1 November 1950, the best fighter the USAF had in the theater was the straight-wing F-80C, and it was unable to match the swept-wing MiG. The response of the USAF, in retrospect, was curious; it sent a wing of quite similar F-84Es. Starting in December 1950, these begin to escort daylight B-29 raids. It was found to be rugged and a stable gun platform, but like the F-80C was easily out-maneuvered by the faster MiG-15. The role of countering the MiG was given to the F-86, and the F-84 subsequently was used as a fighter-bomber and, towards the end of the war, night intruder. Starting in 1951, many F-80C units in Korea converted to the F-84.

In addition to the USAF, the F-84E was used by the air forces of Belgium, France, Netherlands, and Norway. The F-84G was used by these and also by the air forces of Denmark, Greece, Iran, Italy, Portugal, Republic of China, Thailand, Turkey, and Yugoslavia. Taiwanese F-84s engaged in air combat with PRC MiG-15 and MiG-17s during the Second Taiwan Straight Crisis in 1958. Portuguese F-84s saw combat in the Angolan War of Independence from 1961 as fighter-bombers.

ADCs are provided for:
\begin{itemize}
    \item F-84E
    \item F-84G
\end{itemize}