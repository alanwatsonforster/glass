\section*{F-51 Mustang}


The F-51 Mustang fighter was originally designed and built by North American to British specifications. The early versions were aerodynamically superb and performed excellently at low altitudes, but their performance fell off above 15,000 feet as their engines lacked a supercharger. Once this deficiency was remedied, later versions served very successfully as a long-range escort fighter with the USAAF in Europe and the Pacific. The F-51D was the most numerous version.

After WWII and now in the service of the USAF, the F-51 began to be replaced by the F-80, but nevertheless was extensively used in the Korean War as a fighter-bomber, where its endurance and ability to operate from poor airstrips gave it advantages over contemporary jets. Indeed, in the initial stages of the war, several USAF squadrons converted from the F-80 back to the F-51. Its principal disadvantage as a fighter-bomber was the vulnerability of its liquid-cooling system to small arms fire.

The RF-51D was the photo-reconnaissance version. It was equipped with two K-24 cameras in the rear fuselage, one oblique facing left and the other overhead. 

The USAF used the F-51D and RF-51D in the Korean War, although as adequate airfields in the Korean peninsula became available, they were replaced by the F-80C, RF-80C, and other jets. The F-51D was also used in Korea by No. 77 Sqn RAAF until it was replaced by the Meteor F.8 in April 1951, by No. 2 Sqn SAAF until it was replaced by the F-86 in December 1952, and by the RoKAF. They also served in the air forces of many other countries, and in particular saw action in IAF service in both the 1948 and 1956 wars. 

The gun armament of the F-51D is six .50 cal M2 machine guns with about 300 rounds per gun. A typical ground-to-air load in the Korean War would be a pair of 100 lb, 250 lb, or 500 lb bombs or a pair of 110 US gal (750 lb) napalm tanks on the inner wing stations, often combined with six HVAR rockets that could be carried on the outer wing stations. In RAAF service, RP-3 rockets were also used instead of HVAR rockets. The RF-51D is equipped with guns and the inner-wing stations, but not the outer-wing stations for rockets.

In earlier USAAF service in WWII, the F-51D and RF-51D were designated P-51D and F-6D.

ADCs are provided for the:
\begin{itemize}
\item F-51D
\item RF-51D
\end{itemize}

\section*{F4U Corsair}

The F4U Corsair fighter was designed and built by Vought for the USN. It featured superb performance, but its long nose and a cockpit well to the rear meant it was a challenge to land on an aircraft carrier. During its long gestation, it was employed as a land-based fighter with the USMC, but by the end of WWII, it was regarded as the best carrier-based fighter in service. By the time of the Korean War, it had been replaced as a day fighter by the jet-engined F9F Panther, but continued to serve as a fighter-bomber in the USN and USMC.

The F4U-4 is the last version that was constructed during WWII and maintains the original armament of six .50 cal M2 machine guns. The F4U-4B is basically a -4 with four 20 mm M3 cannon substituting the machine guns. These two versions were used in large numbers in the Korean War for close air support and interdiction, both by the USN and USMC. The -4 was preferred for carrier operations, as its guns were easier to service in the confined spaces of the hanger deck of an aircraft carrier, and the -4B tended to be used by land-based USMC squadrons.

The F4U-5 is a post-WWII version with many refinements based on experience with the -4 and maintaining the 20~mm armament of the -4B. For reasons that are not clear to me, it did not see service in the Korean War.

The AU-1 is a dedicated close air support aircraft for the USMC, derived from the FU4-5, but with heavier armor, a simpler supercharger designed for operations at lower altitudes, and additional weapons stations. It entered service in 1952 and saw combat in the Korean War.

The gun armament of the -4 is six .50 cal M2 machine guns with about 400 rounds per gun (400 rounds for the inner two and 375 rounds for the outer one). The -4B, -5, and AU-1 have four 20 mm M3 cannon with 231 rounds per gun. 

A typical air-to-ground armament for the -4 and -4B in the Korean War would be TODO. 

ADCs are provided for the:
\begin{itemize}
\item F4U-4
\item F4U-4B
\item F4U-4P
\item F4U-5
\item F4U-5P
\item F4U-5N
\item F4U-5NL
\item AU-1
\end{itemize}

\section*{B-26 and A-26 Invader and Counter-Invader}

The Douglas B-26 Invader is a bomber and attack aircraft. It entered service in the USAAF before the end of WWII and saw combat in both the European and  Pacific Theaters. It later served with the USAF in the Korean War, during Operation Farm Gate in South Vietnam, and finally flying “Nimrod” interdiction missions over Laos. Many of its missions in Korea, Vietnam, and Laos were flown at night. It was also used by the Armée de l'air in the First Indochina War, the CIA in the Bay of Pigs Invasion, and in small number in many other conflicts in the 1950s and 1960s.

The B-26 was designed with two remote-control turrets, one dorsal and one ventral, similar to the rear turrets of the B-29 and each equipped with two .50 cal M2 machine guns. The turrets were operated by a single gunner, positioned behind the bomb bay, who monitored the sky though large ventral and dorsal windows and aimed both turrets with an periscope sight. The lower turret was removed in many aircraft to give more fuel capacity. In later service, both turrets were removed as defensive guns were not useful for its missions in Vietnam and Laos. 

The B-26 also had a number of different noses, with the most common being the solid nose with eight .50 cal M2 machine guns on the B-26B and the gunless glass nose (which allowed the use of a bombsight) on the B-26C. There was also some variation in guns fitted in the wings.

The B-26K was a rebuilt version, necessary after several earlier aircraft had been lost because of metal fatigue in the main wing spar. It saw combat from 1966 to 1969 with the USAF, flying from Thailand on night-time interdiction missions in Laos. 

The Invader was originally designated A-26. In 1948, it was redesignated B-26, reusing the designation of the earlier B-26 Marauder which by then had left service. In 1966, the B-26K was redesignated A-26A to avoid the perception of a bomber being based in supposedly neutral Thailand.

Typical armament in the Korean War, beyond the guns, was 500 or 1000 lb bombs in the bomb bay and 500 or 1000 lb bombs, 110 gal napalm cans, or HVARs or parachute flares on the wing stations.

Typical armament of Farm Gate B-26s was TODO.

Typical armament of Nimrod B-26Ks was fragmentation and incendiary bombs in the bomb bay, and then a mixture of illumination pods, napalm, LAU-3A rocket pods, and CBUs under the wings.

ADCs are provided for the:
\begin{itemize}
\item B-26B (Two Turrets)
\item B-26B (One Turret)
\item B-26B (No Turrets)
\item B-26C (Two Turrets)
\item B-26C (One Turret)
\item B-26C (No Turrets)
\item B-26K
\item A-26A
\end{itemize}
