\section*{F-51 Mustang}

The F-51 fighter was originally designed and built by North American to British specifications. The early versions were aerodynamically superb and performed excellently at low altitudes, but their performance fell off above 15,000 feet as their engines lacked a supercharger. Once this deficiency was remedied, later versions served very successfully as a long-range escort fighter with the USAAF in Europe and the Pacific. The F-51D was the most numerous version.

After WWII and now in the service of the USAF, the F-51 began to be replaced by the F-80, but nevertheless was extensively used in the Korean War as a fighter-bomber, where its endurance and ability to operate from poor airstrips gave it advantages over contemporary jets. Indeed, in the initial stages of the war, several USAF squadrons converted from the F-80 back to the F-51. Its principal disadvantage as a fighter-bomber was the vulnerability of its liquid-cooling system to small arms fire.

The RF-51D was the photo-reconnaissance version. It was equipped with two K-24 cameras in the rear fuselage, one oblique facing left and the other overhead. 

The USAF used the F-51D and RF-51D in the Korean War, although as adequate airfields in the Korean peninsula became available, they were replaced by the F-80C, RF-80C, and other jets. The F-51D was also used in Korea by No. 77 Sqn RAAF until it was replaced by the Meteor F.8 in April 1951, by No. 2 Sqn SAAF until it was replaced by the F-86 in December 1952, and by the RoKAF. They also served in the air forces of many other countries, and in particular saw action in IAF service in both the 1948 and 1956 wars. 

The gun armament of the F-51D is six .50 cal M2 machine guns with about 300 rounds per gun. A typical ground-to-air load would be a pair of 100 lb, 250 lb, or 500 lb bombs or a pair of 110 US gal (750 lb) napalm tanks on the inner wing stations, often combined in the with six HVAR rockets could be carried on the outer wing stations. In RAAF service, RP-3 rockets were also used instead of HVAR rockets. The RF-51D is equipped with guns and the inner-wing stations, but not the outer-wing stations for rockets.

In earlier USAAF service in WWII, the F-51D and RF-51D were designated P-51D and F-6D.

ADCs are provided for the:
\begin{itemize}
\item F-51D
\item RF-51D
\end{itemize}
