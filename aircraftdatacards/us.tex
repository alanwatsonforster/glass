\section*{F-51 Mustang}

The F-51 fighter was originally designed and built by North American to British specifications during WWII. The early versions were aerodynamically superb and performed excellently at low altitudes, but their performance fell off above 15,000 feet as their engines lacked a supercharger. Once this deficiency was remedied, later versions served very successfully as a long-range escort fighter with the USAAF in Europe and the Pacific. The F-51D was the most numerous version.

After WWII and now in the service of the USAF, the F-51 began to be replaced by the F-80, but nevertheless was extensively used in the Korean War as a fighter-bomber, where its endurance and ability to operate from poor airstrips gave it advantages over contemporary jets. Indeed, in the initial stages of the war, several USAF squadrons converted from the F-80 back to the F-51. Its principal disadvantage as a fighter-bomber was the vulnerability of its liquid-cooling system to small arms fire.

The RF-51D was the photo-reconnaissance version. It was equipped with two K-24 cameras in the rear fuselage, one oblique facing left and the other overhead. 

The USAF used the F-51D and RF-51D in the Korean War, although as adequate airfields in the Korean peninsula became available, they were replaced by the F-80C, RF-80C, and other jets. The F-51D was also used in Korea by No. 77 Sqn RAAF until it was replaced by the Meteor F.8 in April 1951, by No. 2 Sqn SAAF until it was replaced by the F-86 in December 1952, and by the RoKAF. They also served in the air forces of many other countries, and in particular saw action in IAF service in both the 1948 and 1956 wars. 

The gun armament of the F-51D is six .50 cal M2 machine guns with about 300 rounds per gun. A typical ground-to-air load would be a pair of 100 lb, 250 lb, or 500 lb bombs or a pair of 110 US gal (750 lb) napalm tanks on the inner wing stations, often combined in the with six HVAR rockets could be carried on the outer wing stations. In RAAF service, RP-3 rockets were also used instead of HVAR rockets. The RF-51D is equipped with guns and the inner-wing stations, but not the outer-wing stations for rockets.

In earlier USAAF service in WWII, the F-51D and RF-51D were designated P-51D and F-6D.

ADCs are provided for the:
\begin{itemize}
\item F-51D
\item RF-51D
\end{itemize}

\section*{F4U and AU Corsair}

The F4U fighter was developed by Vought for the USN before and during WWII. However, due to difficulties in carrier landings and the desire to simplify the supply of parts to aircraft carriers in the Pacific Theater, it was used initially by the USMC as a land-based fighter in the Solomons campaign and by the RN. It was used on carriers by the USN only towards the end of WWII.

After WWII, it was replaced as a fighter by the F9F Panther and F2H Banshee, but served in the Korean War with both the USN and USMC as a light fighter bomber alongside the heavier AD Skyraider.

The F4U-4 and F4U-4B were produced at the end of WWII and differed largely in their gun armament, with the -4 having six .50 cal M2 machine guns each with about 400 rounds and the -4B having four 20 mm M3 cannon each with about 230 rounds. They were equipped with two inner wing stations for bombs or fuel tanks and, initially, eight rocket launchers on the outer part of the wing. These were later adapted or replaced to allow the carriage of bombs and flares. Typical loads might be 500 or 1000 lb bombs on the inner wing stations, sometimes mixing a 150 US gal fuel tank on one station with a bomb on the other, and eight 100 or 250 lb bombs, eight HVAR or ATAR rockets, or four 500 lb bombs on the outer wing stations.

The F4U-5 entered service with the USN after WWII, but did not see service in the Korean War.

The F4U-4P and -5P were both photographic reconnaissance aircraft, essentially the base -4 or -5 with a camera that could be mounted in an oblique or overhead configuration. Both saw service in the Korean War with the USN and USMC.

The F4U-5N was a night fighter and night attack version of the -5, with an APS-19 radar in a pod on its starboard wing. The -5NL was a winterized version of the -5N. Both saw service in the Korean War with the USN and USMC.

The AU-1 was a specialized close-support aircraft developed for the USMC taking into account early experience in the Korean War. It featured additional armor, oil coolers relocated to reduce vulnerability, simplified superchargers, and two additional weapon stations. It saw service in the Korean War from 1952.

ADCs are provided for the:
\begin{itemize}
\item F4U-4
\item F4U-4B
\item F4U-4P
\item F4U-5
\item F4U-5P
\item F4U-5N
\item F4U-5NL
\item AU-1
\end{itemize}
