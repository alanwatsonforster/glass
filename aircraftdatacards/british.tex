\section*{Meteor}

The Gloster Meteor was the first British jet fighter and the only Allied jet to see combat in WW2. It entered service as a day fighter with the RAF in 1944 and continued to serve in various roles after the war. It was powered by two wing-mounted centrifugal flow jet engines, had unswept wings, and was armed with four 20 mm guns.

The F.8 was a post-WW2 development. Compared to its immediate predecessor, the late-WW2 F.4, it had more powerful Rolls-Royce Derwent 8 engines, a lengthened fuselage, and a new tail to solve center-of-gravity problems, improved visibility from the cockpit, and the capability to carry air-to-ground weapons. In common with many early jet fighters, it suffered from shorter ranges, and this was addressed by equipping it with a jettisonable, conformal ventral fuel tank. It was introduced into RAF service in 1949. In the 1950s, the unsophisticated aerodynamics of the Meteor led it to be outclassed by newer swept-wing fighters like the MiG-15 and F-86, and so the F.8 was the last fighter version produced. Subsequent versions were reconnaissance and night fighter.
 
The FR.9 was a photoreconnaissance version of the F.8. It had an extended nose for a single camera and retained the full combat capability of the F.8. It was introduced into RAF service in 1950.

The F.8 served as the main RAF fighter and fighter-bomber in the 1950s. It was also extensively used by No.~77 Squadron RAAF in the Korean War, replacing their F-51Ds in April 1951. In the Suez crisis, the F.8 was used by the Egyptian, Syrian, and Israeli air forces, and the FR.9 by the RAF. The F.8 or FR.9 also served in the air forces of Belgium, Brazil, Denmark, Ecuador, and the Netherlands. The Meteor was replaced in RAF service by the Canadair Sabre 4, Hawker Hunter, and Gloster Javelin and in RAAF service by the Avon Sabre.

% Sabre 4 in RAF?
% Other users?
% Add PR.10?

A typical air-to-air load would be the ventral 175 gal fuel tank, perhaps with two 100 gal fuel tanks on the wings to increase patrol time. For air-to-ground, the Meteor could carry two 1000 lb bombs or eight or sixteen RP-3 rockets in addition to the ventral tank.

ADCs are provided for the:
\begin{itemize}
\item Meteor F.8
\item Meteor FR.9
\end{itemize}

\section*{Sea Fury}

The Hawker Sea Fury was a post-WW2 propeller-driven fighter bomber. 

The Sea Fury was a descendant of the WW2 Hawker Tempest fighter bomber, originally adapted as a long-range fighter for use in the war against Japan. After WW2 ended, the RAF no longer had interest, but the RN acquired a version adapted for carrier operations as a replacement for its Seafires, which were not well suited to carrier operations, and Corsairs, which had to be returned to the US as the Lend-Lease program ended. At the time, the high landing speeds of the first generation of jet aircraft was an impediment to their use on carriers. The Sea Fury was powered by a Centaur radial engine, armed with four 20 mm guns, and had a bubble canopy with an excellent view except under the nose.

There were three major versions of the Sea Fury. The first was the F.10 day fight, which entered service with the RN in 1947. This was quickly followed by the FB.11, which added armor and weapon stations to fulfill the fighter-bomber role better. In 1950, a two-seater T.20 trainer was deployed. The Sea Fury F.50, FB.50, and FB.51 and the Fury FB.60 were export versions of the F.10 and FB.11 with minor changes and removal of carrier-specific equipment from the FB.60. 

The Sea Fury was exported to Australia, Burma, Canada, Cuba, Iraq, the Netherlands, and Pakistan. The RN and RAN used it as a fighter-bomber in the Korean War and also saw combat with the Cuban Revolutionary Air Force during the Bay of Pigs invasion and with the Netherlands Royal Navy in the Dutch East Indies. It was replaced in the RN by the Sea Hawk and Attacker starting in 1953, in the RCN by the F2H Banshees starting in 1956, and in the Netherlands Royal Navy by Sea Hawks.

A typical air-to-ground load would be two 500- or 1000-lb bombs or twelve RP-3 rockets. It could also carry two 90-gallon fuel tanks to extend its range. Alternatively, it could be equipped with cameras for photographic reconnaissance missions.

An ADC is provided for the:
\begin{itemize}
\item Sea Fury FB.10
\end{itemize}
